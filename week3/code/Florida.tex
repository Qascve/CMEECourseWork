\documentclass[conference]{IEEEtran}
\usepackage[a4paper,margin=1in]{geometry}
\usepackage{graphicx}
\usepackage{amsmath}

\title{Florida Correlation Result}
\author{}
\date{}

\begin{document}

\maketitle

\section*{Result Explanation}

The analysis shows a positive correlation between the two variables, with $p = 0.5332$.
The $p$-value is $0$, which means the result is statistically significant.
In other words, the observed relationship is very unlikely to have happened by chance.

\begin{figure}[]
  \centering
  \includegraphics[width=0.4\textwidth]{../results/Florida_analysis1.pdf}
  \caption{Key West annual temperature trends with linear regression (1901-2000)}
\end{figure}


\begin{figure}[h!]
  \centering
  \includegraphics[width=0.4\textwidth]{../results/Florida_analysis2.pdf}
  \caption{Distribution of random correlations from permutation test}
\end{figure}


\begin{figure}[h!]
  \centering
  \includegraphics[width=0.4\textwidth]{../results/Florida_analysis3.pdf}
  \caption{Temperature changes relative to 1901-1905 baseline}
\end{figure}

\end{document}
